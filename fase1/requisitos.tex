\documentclass{article}
\usepackage[utf8]{inputenc}

\title{Primeira fase do Projeto}
\author{Gustavo Silva, Leonardo Padilha, \\Lucas Santos, Marcos Kawakami, \\Matheus Oliveira e Victor Sena Molero}
\date{}

\begin{document}

\maketitle

\section{Objetivo}
Este sistema tem objetivo de auxiliar o gerenciamento de projetos. Como este objetivo em mente fazemos o uso de segmentar estes em tarefas com o intuito de facilitar o controle e melhorar a modularização.

\section{Funcionalidades do Gerenciador de Projetos}
\begin{itemize}
    \item Criar/Remover um projeto
    \\Todo projeto pode ser criado ou removido. Projetos podem ser removidos somente pelos usuários que estão no grupo de administradores deste, e podem ser criados e neste momento um grupo de usuários administradores é determinado (e pode ser futuramente modificado, por qualquer membro deste grupo).
    \item Organização das tarefas
    \\Todo projeto é composto de uma ou mais tarefas, que tem um grupo não vazio de usuários que a administram. Cada usuário pode fazer parte de mais de um grupo de administradores de tarefas. Cada tarefa tem associado a ela uma data de início e de limite.
\end{itemize}

\section{Requisitos de Dados}
\begin{itemize}
    \item Usuário
    \\ Todo usuário possui um e-mail registrado que o representa univocamente e é utilizado para acessar o sistema, um inteiro (identificador), uma lista dos projetos que participa e uma lista do projetos em que é administrador.
    %talvez um inteiro com algum começo, ou talvez o hash do email.
    \item Projeto
    \\Todo projeto possui um título, e é identificada por um inteiro que o representa univocamente.
    %aqui q rola o problema do inteiro, q pode confundido com o do usuário
    \item Tarefa
    \\Toda tarefa possui data de início e data limite, e é identificada por um inteiro.
    \item Administrador
    \\Todo projeto e tarefas possuem administradores que são grupos de usuários que ter permissões.
    %precisa explicar melhor
    \item Data limite
    \\Toda tarefa possui uma data limite
\end{itemize}
\section{Requisitos funcionais}
\begin{itemize}
    \item Usuário
    \begin{itemize}
        \item Uma pessoa pode se cadastrar no sistema de gerenciamento
        \item Um e-mail pode estar vinculado a somente uma pessoa cadastrada.
    \end{itemize}
    \item Projeto
    \begin{itemize}
        \item Cada usuário pode criar uma quantidade ilimitada de projetos
        \item O usuário criador do projeto faz parte por padrão do grupo de administradores (porém pode sair desde caso queira)
    \end{itemize}
    \item Tarefa
    \begin{itemize}
        \item Qualquer usuário de um projeto pode criar tarefas
        \item Somente administradores da tarefa podem alterar datas.
    \end{itemize}
    \item Administrador
    \begin{itemize}
        \item Pode alterar qualquer aspecto da tarefa ou projeto que administra, exceto remover outro administrador.
    \end{itemize}
    \item Data limite
    \begin{itemize}
        \item Uma vez que uma tarefa é criada, seu criador determina uma data limite
        \item Uma vez determinada a data limite, esta só pode ser alterada por um administrador da tarefa relacionada ou do projeto na qual a tarefa faz parte.
    \end{itemize}
\end{itemize}
\end{document}

