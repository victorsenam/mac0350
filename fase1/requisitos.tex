\documentclass{article}
\usepackage[utf8]{inputenc}

\title{Primeira fase do projeto de MAC0350}
\author{Gustavo Silva \\Leonardo Padilha \\Lucas Santos \\Marcos Kawakami - 8041331 \\Matheus Oliveira - 8642821 \\Victor Sena Molero - 8941317}
\date{}

\begin{document}

\maketitle

\section{Objetivo}
Este sistema tem objetivo de auxiliar o gerenciamento de projetos. Para facilitar o controle e melhorar a modularização do projeto, estes são segmentados em tarefas.

\section{Funcionalidades do gerenciador de projetos}
\begin{itemize}
    \item Gerenciamento de projeto
    \\Todo projeto pode ser criado ou removido por um usuário. O usuário que cria o projeto é seu administrador. Cada projeto pode ser removido pelo seu administrador. Cada projeto possui um ou mais usuários. O administrador pode adicionar ou remover usuários do projeto e faz parte do conjunto de usuários do projeto.
    \item Gerenciamento das tarefas
    \\Um projeto possui tarefas. Cada tarefa deve possuir um ou mais usuários responsáveis. Um usuário pode ser responsável por várias tarefas. Cada tarefa possui uma data de início e uma data de finalização. Uma tarefa pode ser criada por qualquer usuário do projeto.
\end{itemize}

\section{Requisitos de dados}
\begin{itemize}
    \item Usuário
    \\ Todo usuário possui um e-mail registrado que o representa univocamente e é utilizado para acessar o sistema, um inteiro (identificador), uma lista dos projetos que participa, uma lista do projetos em que é administrador e uma senha.
    \item Projeto
    \\Todo projeto possui um título, um inteiro (identificador), uma lista de tarefas, um usuário administrador e um conjunto de usuários.
    \item Tarefa
    \\Toda tarefa possui data de início e de finalização, um inteiro (identificador) e uma lista de usuários responsáveis.
\end{itemize}
\section{Requisitos funcionais}
\begin{itemize}
    \item Usuário
    \begin{itemize}
        \item Uma pessoa pode se cadastrar no sistema de gerenciamento;
        \item O usuário pode desativar seu cadastro.
    \end{itemize}
    \item Projeto
    \begin{itemize}
        \item Cada usuário pode criar uma quantidade ilimitada de projetos;
        \item O usuário criador do projeto é administrador do projeto;
        \item O usuário pode convidar usuários a participarem de um projeto do qual é administrador;
        \item O usuário pode aceitar ou recusar um convite a participar de um projeto;
        \item Os usuários de um projeto podem criar tarefas naquele projeto.
    \end{itemize}
    \item Tarefa
    \begin{itemize}
        \item Qualquer usuário do projeto a qual pertence a tarefa pode editar e remover a tarefa.
    \end{itemize}
\end{itemize}
\end{document}

